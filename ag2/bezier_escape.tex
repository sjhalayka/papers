\documentclass[12pt]{article}

\title{A 2nd note on anisotropic quantum gravity}
\author{S. Halayka\footnote{sjhalayka@gmail.com}}
\date{\today\;\currenttime}

\usepackage{datetime}
\usepackage{listings}
\usepackage{cite}
\usepackage{xcolor}
\usepackage{graphicx}
\usepackage{setspace}
\usepackage{amsmath}
\usepackage{url}
\usepackage[margin=1in]{geometry}
%\doublespace



\begin{document}



 
\maketitle

\begin{abstract}
In a previous paper, we introduced a model that accounts for both dark matter and dark energy. In this paper we will attempt to refute objections to the model.
\end{abstract}


\section{Cold dark matter from a graviton condensate}

We assume that a lone graviton propagates at the speed of light. That is, without any {\textit{relaying}}, the graviton travels at the speed of light.

One objection to the model introduced in \cite{halayka} is that dark matter must be cold. In other words, the dark matter must have a speed less than the speed of light.

It is easy to see that once the gravitational degrees of freedom are aligned, that there will be graviton-graviton interaction -- gravitons relaying other gravitons. This relaying only slows the graviton down, making it cold. As the spatial dimension of the gravitationally bound system drops from 3 down to 2, then down to 1, the slower the gravitons.

Thanks to W. for their objection.




\begin{thebibliography}{9}

\bibitem{halayka} Halayka. A note on anisotropic quantum gravity.\\TechRxiv. Preprint -- https://doi.org/10.36227/techrxiv.20326470.v5


\end{thebibliography}





\end{document}









