\documentclass[12pt]{article}

\title{A 2nd note on anisotropic quantum gravity}
\author{S. Halayka\footnote{sjhalayka@gmail.com}}
\date{\today\;\currenttime}

\usepackage{datetime}
\usepackage{listings}
\usepackage{cite}
\usepackage{xcolor}
\usepackage{graphicx}
\usepackage{setspace}
\usepackage{amsmath}
\usepackage{url}
\usepackage[margin=1in]{geometry}
%\doublespace



\begin{document}



 
\maketitle

\begin{abstract}
In a previous paper, we introduced a model that accounts for both dark matter and dark energy. In this paper we will attempt to refute objections to the model.
\end{abstract}


\section{On cold dark matter from a graviton condensate}

One objection to the model introduced in \cite{halayka} is that dark matter must be cold. 
In other words: the dark matter must have a speed much less than the speed of light in vacuum.

We assume that a lone graviton propagates at the speed of light. 
That is, without any {\textit{relaying}}, the graviton travels at the speed of light.

Gravitons at least undergo Shapiro delay in vacuum -- a graviton in the presence of other gravitons travels with a speed less than the speed of light.
The speed of the graviton can only be further slowed down when travelling through a mass.

This model is experimentally verifiable. 
There will be no gravitational shadow behind a mass, but there will be a lag -- the gravitons travel slower than the speed of light while being relayed by a mass.






\section{On Loop Quantum Gravity}

One objection is that the model is not compatible with Loop Quantum Gravity.

The model shows that there is a tetrahedral substratum underlying the 4 known interactions.
The only difference is that this model predicts that these tetrahedra will not be as tiny as the Planck scale, making them all that much easier to experimentally verify.







\begin{thebibliography}{9}

\bibitem{halayka} Halayka. A note on anisotropic quantum gravity.\\TechRxiv. Preprint -- https://doi.org/10.36227/techrxiv.20326470.v5


\end{thebibliography}





\end{document}









