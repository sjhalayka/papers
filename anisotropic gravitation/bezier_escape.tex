\documentclass[12pt]{article}

\title{A note on anisotropic quantum gravity}
\author{S. Halayka\footnote{sjhalayka@gmail.com}}
\date{\today\;\currenttime}

\usepackage{datetime}
\usepackage{listings}
\usepackage{cite}
\usepackage{xcolor}
\usepackage{graphicx}
\usepackage{setspace}
\usepackage{amsmath}
\usepackage{url}
\usepackage[margin=1in]{geometry}
%\doublespace



\begin{document}



 
\maketitle

\begin{abstract}
Anisotropic gravitation is considered, leading to a unique view of dark matter.
\end{abstract}






\section{On the interruption of processes by time dilation}

A {\textit{process}} is a system of mass-energy, including its internal interactions.

Time dilation is the {\textit{interruption}} of said process, whether it be kinematic and/or gravitational -- both are the result of external interactions.
In the case of the gravitational interaction, the process is interrupted by spacetime itself (e.g. gravitons).
In the case of the non-gravitational interaction, the process is interrupted by the other force-carrying particles (e.g. photons, etc).

It is the gradient of the gravitational time dilation \cite{misner}, based on the distance from the gravitating process, that causes acceleration toward the gravitating process.
This is encoded in the first term on the right-hand side of the Schwarzschild line element
\begin{equation}
ds^2 = -\left( 1 - \frac{R_s}{r} \right) dt^2 + \frac{dr^2}{\left( 1 - \frac{R_s}{r} \right)} + r^2 (d\theta^2 + \sin^2 \theta d\phi^2).
\end{equation}
where $R_s$ is the Schwarzschild radius
\begin{equation}
R_s = 2M,
\end{equation}
and $M$ is the mass-energy of the gravitating process, in natural (Planck) units, where $c = G = \hbar = k_b =	 \ell_p = 1$.

If all of physics is about processes, then it is therefore all about {\textit{computation}} \cite{zuse, wolfram} -- here we have even adopted the concept of process interruption, which is surely familiar to all assembly programmers \cite{abrash}.

The densest process for any given amount of mass-energy $M$ is a black hole -- contemporary digital or quantum processors are nowhere close to this limit.

In Newton and Einstein's theory, all mass-energy gravitates in an {\textit{isotropic}} (spherical) manner.
In this paper, we will consider aspherical -- {\textit{anisotropic}} -- gravitating processes.







\section{On anti-gravity and the overclocking of processes}

It should be noted that there is no {\textit{anti-gravity}} -- there is no {\textit{anti-interruption}.
That said, there is the {\textit{overclocking}} and {\textit{optimization}} of processes to consider -- literally making a process run faster than its natural rate by eliminating redundant interactions.








\section{On taking the holographic principle literally}

It takes $n$ Boolean {\textit{degrees of freedom}} to describe the gravitational field \cite{hooft, susskind, bousso} generated by a process of mass-energy $M$.
This is regardless of the radius of the gravitating process, and regardless of how many non-gravitational degrees of freedom exist.
The number of gravitational degrees of freedom is
\begin{equation}
n = \frac{A_s}{4 \log 2},
\end{equation}
where in the case of the Schwarzschild line element
\begin{equation}
A_s = 4 \pi R_s^2.
\end{equation}

In the case where the process is a black hole, this effectively quantizes the event horizon -- the event horizon is made up of an ensemble of Planckian oscillators.
All of the non-gravitational degrees of freedom have been stripped away as gravitational waves, leaving only the gravitational degrees of freedom.
In other words: a black hole is raw spacetime.

Note that as a process falls toward a black hole's event horizon, the process is interrupted to the point where it becomes fully {\textit{assimilated}} -- it becomes one process with the black hole.








\section{On connectionless versus connected interaction}

Unlike long-range electromagnetism, which is a {\textit{connectionless}} interaction, gravitation is a {\textit{connected}} interaction.

For electromagnetism, the photons are `fired and forgotten' -- there is only one oscillator per non-gravitational degree of freedom: the sender.

On the other hand, for gravitation, which is a connected interaction, there are two oscillators per gravitational degree of freedom: there is the sender and the receiver (sometimes kiloparsecs apart).
This is a kind of gravitational entanglement between the sender and receiver, a relic from the beginning of the Big Bang that persists to this day.
For gravitation, there are $n$ concurrent connections per process -- no more, no less.
Mach's principle \cite{misner} is validated.

To use another analogy from computer science \cite{stevens}, the connectionless interaction is like a User Datagram Protocol (UDP) network socket, and the connected interaction is like a Transmission Control Protocol (TCP) network socket.
Of course, higher-level connections exist in the case of electromagnetism, such as in fibre optics.

It's important to note that there is no such thing as a gravitational shadow.




\section{On dark matter}

With regard to the flat rotation curve found in galactic dynamics \cite{binney}: if this number $n$ is at least conserved as a gravitationally bound process (e.g. a galaxy) goes from sphere to disk as distance from the process centre increases, then the gravitation becomes anisotropic, strengthening along the orbit plane, weakening elsewhere.
In fact, gravitation is anisotropic for all gravitationally bound processes, for there is no such thing as a perfect spherically symmetric, isotropic, homogeneous process (not even a black hole is spherically symmetric, because the event horizon is quantized).
This includes galaxies, clusters, walls, and filaments.
For instance, for a perfect disk, the interaction strength increases by a factor of $c$ with a long-range falloff proportional to $1/r$, and for a perfect filament it increases by a factor of $c^2$ with a long-range falloff proportional to $1$ (e.g. no falloff).
For more details, see \cite{halayka}, where we show that for these perfect shapes that the gravitational field goes from being ($3+1$)-dimensional down to ($2+1$) or ($1+1$)-dimensional.
For example, at a distance of $10$ kiloparsecs from the Galactic centre, it is found that the dimension of the gravitational field is roughly $(2.96 + 1)$.

We have no reason to expect that we will find a WIMP, axion, or similar solution \cite{berger, aalbers, quiskamp, haipeng, hui, ackerman} to the dark matter problem -- if gravitation is quantized, then dark matter is made up of a graviton {\textit{condensate}}.

Without evidence to the contrary, it is assumed that the maximum speed of the graviton is the same as the maximum speed of the photon (e.g., $1$ Planck unit of length per Planck unit of time).






\section{Conclusion}

In this paper we have defined a unique view of dark matter, which forms due to anisotropic gravitation in gravitationally bound processes.
Of greatest importance is the fact that there is a finite number of gravitational degrees of freedom $n$ for a process of mass-energy $M$, and that when aligned, these gravitational degrees of freedom form gravitational bonds that are stronger than those predicted by Newton and Einstein's isotropic theory of gravitation.

It should be noted that for processes bound by all four forces, such as protoplanetary disks, there is practically no dark matter to be found, because the emission of gravitons is so very close to being isotropic due to the isotropic nature of the other three forces.
For the Solar system -- which was once a protoplanetary disk -- the overall mass-energy distribution has become too sparse to form any appreciable amount of anisotropic gravitation in the Sun.
This is to say that Newton and Einstein's isotropic theory of gravitation is still very great, for the AU-scale cases where dark matter need not be factored in.

It is expected that gravitational wave detectors will find evidence of dark matter \cite{LIGO}, given that gravitational waves are composed of gravitons.




\section{Data availability statement (DAS)}

No data were produced for this paper.



\pagebreak






\begin{thebibliography}{9}
\bibitem{misner} Misner et al. Gravitation. (1970)
\bibitem{zuse} Zuse. Calculating Space. (1969)
\bibitem{wolfram} Wolfram. A New Kind of Science. (2002)
\bibitem{abrash} Abrash. Michael Abrash's Graphics Programming Black Book. (1997)
\bibitem{hooft} `t Hooft. Dimensional reduction in quantum gravity. (1993)
\bibitem{susskind} Susskind. The World as a Hologram. (1994)
\bibitem{bousso} Bousso. The holographic principle. (2002)

\bibitem{stevens} Stevens et al. Unix Network Programming, Volume 1: The Sockets Networking API. (2003)

\bibitem{binney} Binney et al. Galactic Dynamics. Second Edition. (2008)
\bibitem{halayka} Halayka. Is the anisotropic interaction of luminous matter responsible for the extrinsic gravitation usually attributed to exotic dark matter? (2008)

\bibitem{berger} Berger et al. Snowmass 2021 White Paper: Cosmogenic Dark Matter and Exotic Particle Searches in Neutrino Experiments. (2022)
\bibitem{aalbers} Aalbers et al. First Dark Matter Search Results from the LUX-ZEPLIN (LZ) Experiment. (2022)
\bibitem{quiskamp} Quiskamp et al. Direct Search for Dark Matter Axions Excluding ALP Cogenesis in the 63-67 micro-eV Range, with The ORGAN Experiment. (2022)
\bibitem{haipeng} Haipeng et al. Direct detection of dark photon dark matter using radio telescopes. (2022)
\bibitem{hui} Hui et al. Ultralight scalars as cosmological dark matter. (2016)
\bibitem{ackerman} Ackerman et al. Dark Matter and Dark Radiation. (2008)

\bibitem{LIGO} The LIGO Scientific Collaboration, et al. Constraints on dark photon dark matter using data from LIGO's and Virgo's third observing run. (2022)


\end{thebibliography}





\end{document}









