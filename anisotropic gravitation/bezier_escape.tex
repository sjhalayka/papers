\documentclass[12pt]{article}

\title{On anisotropic gravitation}
\author{S. Halayka\footnote{sjhalayka@gmail.com}}
\date{\today}

\usepackage{listings}
\usepackage{cite}
\usepackage{xcolor}
\usepackage{graphicx}
\usepackage{setspace}
\usepackage{amsmath}
\usepackage{url}
\usepackage[margin=1in]{geometry}
%\doublespace



\begin{document}



 
\maketitle

\begin{abstract}
Anisotropic gravitation is considered, leading to a unique view of dark matter.
\end{abstract}






\section{On processes}

A {\textit{process}} is a system of mass-energy, along with its internal interactions.

Time dilation is the {\textit{interruption}} of said process, whether it be kinematic and/or gravitational -- both are the result of external interactions.
In the case of the gravitational interaction, the process is interrupted by spacetime itself (gravitons).
In the case of the non-gravitational interaction, the process is interrupted by the other force-carrying particles (e.g. photons, etc).

It is the gradient of the gravitational time dilation, based on the distance from the gravitating process, that causes acceleration toward the gravitating process.
Thus, all of physics is about processes and {\textit{computation}} -- we even adopt the concept of process interruption, which is familiar to all assembly programmers.

Note that the densest process for any given amount of mass-energy is a black hole.
As a process falls toward a black hole's event horizon, the process is interrupted to the point where it becomes fully {\textit{assimilated}} -- it becomes one process with the black hole.

In Newton and Einstein's theory, all mass-energy gravitates in an {\textit{isotropic}} (spherical) manner.
In this paper, we will consider aspherical -- {\textit{anisotropic}} -- gravitating processes.







\section{On the holographic principle}

It takes $n$ Boolean degrees of freedom to describe the gravitational field \cite{hooft, susskind} generated by a process of mass-energy $M$.
This is regardless of the radius of the gravitating process, and regardless of how many non-gravitational degrees of freedom exist.
Where $k_b$ is the Boltzmann constant, the number of gravitational degrees of freedom is
\begin{equation}
n = \frac{k_b \pi R_s^2}{ \ell_p^2 \log 2}.
\end{equation}
Note that this effectively quantizes the event horizon.
The Schwarzschild radius where the event horizon lies at is
\begin{equation}
R_s = \frac{2GM}{c^2},
\end{equation}
and the Planck length is
\begin{equation}
\ell_p = \sqrt{\frac{\hbar G}{c^3}}.
\end{equation}

In the case where the process is a black hole, all of the non-gravitational degrees of freedom have been stripped away as gravitational waves, leaving only the gravitational degrees of freedom.
In other words: a black hole is raw spacetime.






\section{On dark matter}

With regard to the flat rotation curve found in galactic dynamics: if this number $n$ is at least conserved as a gravitationally bound process (e.g. a galaxy) goes from sphere to disk as distance from the process centre increases, then the gravitation becomes anisotropic, strengthening along the orbit plane, weakening elsewhere.
In fact, gravitation is anisotropic for all gravitationally bound processes, for there is no such thing as a perfect spherically symmetric, isotropic, homogeneous process (not even a black hole is spherically symmetric, because the event horizon is quantized).
This includes galaxies, clusters, walls, and filaments.
For instance, for a perfect disk, the interaction strength increases by a factor of $c$, and for a perfect filament it increases by a factor of $c^2$.
For more details, see \cite{halayka}, where we show that for these perfect shapes that the gravitational field goes from being ($3+1$)-dimensional down to ($2+1$) or ($1+1$)-dimensional.
For example, at a distance of $10$ kiloparsecs from the Galactic centre, it is found that the dimension of the gravitational field is roughly $(2.96 + 1)$.

We have no reason to expect that we will find a WIMP, axion, or similar solution \cite{berger, aalbers, quiskamp, haipeng, hui} to the dark matter problem -- if gravitation is quantized, then dark matter is made up of a graviton condensate.

For processes bound by all four forces, such as protoplanetary disks, there is practically no dark matter to be found, because the emission of gravitons is close to isotropic.




\section{Conclusion}

In this paper we have defined a unique view of dark matter, which forms due to anisotropic gravitation in gravitationally bound processes.
In essence, dark matter is a graviton condensate.

Of greatest importance is the fact that there is a finite number of gravitational degrees of freedom, and that when aligned, they form gravitational bonds stronger than that predicted by Newton and Einstein's theory of gravitation.
The key to this is anisotropic gravitation.







\pagebreak





\begin{thebibliography}{9}
\bibitem{hooft} Hooft. Dimensional reduction in quantum gravity. (1993)
\bibitem{susskind} Susskind. The World as a Hologram. (1994)
\bibitem{halayka} Halayka. Is the anisotropic interaction of luminous matter responsible for the extrinsic gravitation usually attributed to exotic dark matter? (2008)
\bibitem{berger} Berger et al. Snowmass 2021 White Paper: Cosmogenic Dark Matter and Exotic Particle Searches in Neutrino Experiments. (2022)
\bibitem{aalbers} Aalbers et al. First Dark Matter Search Results from the LUX-ZEPLIN (LZ) Experiment. (2022)
\bibitem{quiskamp} Quiskamp et al. Direct Search for Dark Matter Axions Excluding ALP Cogenesis in the 63-67 micro-eV Range, with The ORGAN Experiment. (2022)
\bibitem{haipeng} Haipeng et al. Direct detection of dark photon dark matter using radio telescopes. (2022)
\bibitem{hui} Hui, et al. Ultralight scalars as cosmological dark matter. (2016)
\end{thebibliography}





\end{document}









