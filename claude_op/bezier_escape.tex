\documentclass[12pt]{article}

\title{Euclidean cross product in dimension $n \ge 3$ as applied to the matrix determinant}
\author{S. Halayka\footnote{sjhalayka@gmail.com}}
\date{\today\;\currenttime}

\usepackage{datetime}
\usepackage{listings}
\usepackage{cite}
\usepackage{xcolor}
\usepackage{graphicx}
\usepackage{setspace}
\usepackage{amsmath}
\usepackage{url}
\usepackage[margin=0.8in]{geometry}
\usepackage{listings}
\usepackage{xcolor}



\lstset { %
    language=C++,
    backgroundcolor=\color{black!5}, % set backgroundcolor
    basicstyle=\ttfamily\footnotesize,% basic font setting
    showstringspaces=false,
}


%\doublespace

%\usepackage[]{lineno}
%\linenumbers


\begin{document}



 
\maketitle

\begin{abstract}
This paper contains a short introduction to the Claude cross product operator in dimension $n \ge 3$.
The main focus is on some C++ code.
\end{abstract}




\section{Application: the matrix determinant}

In this paper, we focus on the Claude cross product operator in dimension $n \ge 3$.
We refer to this as the Claude cross product operator because the beginnings of this paper/code were supplied by the Claude AI.
Various changes to the code were needed in order for it to operate properly, but the kernel of the idea remains the same.
I've asked Claude if this is the Hodge star operator or the wedge operator, but Claude says no.
The chat log is at:

\url{https://claude.ai/chat/3caf4077-28b5-497f-b704-1b0c336a104d}

The main goal is to acquaint the coder with the basic idea behind the Claude cross product operator in $n$-D. 
The operator accepts $(n - 1)$ $n$-vectors as input, and outputs one $n$-vector.
For instance, where $n = 3$, the $3$-D cross product accepts $(n - 1) = $ two $3$-vectors as input.
Using C++ templates, the abstraction to any $n \ge 3$ is provided.
This cross product is used to calculate the matrix determinant.

\section{Code}

Here we include the Eigen linear algebra library, as well as various parts of the standard library:
\begin{lstlisting}
#include <Eigen/Dense>
using namespace Eigen;

#include <iostream>
#include <vector>
#include <numeric>
#include <string>
#include <sstream>
#include <algorithm>
#include <array>
using namespace std;
\end{lstlisting}

Here we define the vector class, where the data type is T (e.g., double), and N is the dimension:
\begin{lstlisting}
template<class T, size_t N>
class Vector_nD
{
public:
  array<T, N> components;

  // Helper function to get the sign of permutation
  static signed char permutation_sign(const array<int, (N - 1)>& perm)
  {
    bool sign = true;

    for (int i = 0; i < (N - 2); i++)
      for (int j = i + 1; j < (N - 1); j++)
        if (perm[i] > perm[j])
          sign = !sign;

    if (sign)
      return 1;
    else
      return -1;
  }

  Vector_nD(const array<T, N>& comps) : components(comps)
  {
  }

  Vector_nD(void)
  {
    components.fill(0.0);
  }

  T operator[](size_t index) const
  {
    return components[index];
  }

\end{lstlisting}

Here we make a static cross product function that takes in $(n - 1)$ $n$-vectors.
This function returns one $n$-vector:
\begin{lstlisting}

  // Claude cross product operator
  static Vector_nD cross_product(const vector<Vector_nD<T, N>>& vectors)
  {
    if (vectors.size() != (N - 1))
    {
      cout << "nD cross product requires (n - 1) input vectors" << endl;
      return Vector_nD<T, N>();
    }

    array<T, N> result;

    for (size_t i = 0; i < N; i++)
      result[i] = 0.0;

    // These are the indices we'll use for each component calculation
    array<int, (N - 1)> base_indices;

    for (int i = 0; i < (N - 1); i++)
      base_indices[i] = i;

    // Skip k in our calculations - 
    // this is equivalent to removing the k-th column
    // For each permutation of the remaining (N - 1) indices
    for (int k = 0; k < N; k++)
    {
      do
      {
        // Calculate sign of this term
        const signed char sign = permutation_sign(base_indices);

        // Calculate the product for this permutation
        T product = 1.0;
        ostringstream product_oss;

        for (int i = 0; i < (N - 1); i++)
        {
          const int col = base_indices[i];

          // Adjust column index if it's past k
          int actual_col = 0;

          if (col < k)
            actual_col = col;
          else
            actual_col = col + 1;

          product_oss << "v_{" << i << actual_col << "} ";

          product *= vectors[i][actual_col];
        }

        if (sign == 1)
          cout << "x_{" << k << "} += " << product_oss.str() << endl;
        else
          cout << "x_{" << k << "} -= " << product_oss.str() << endl;

        result[k] += sign * product;

      } while(next_permutation(
          base_indices.begin(), 
          base_indices.end()));
    }

    // Flip handedness
    for (size_t k = 0; k < N; k++)
      if (k % 2 == 1)
        result[k] = -result[k];

    cout << endl;

    for (int k = 0; k < N; k++)
      cout << "result[" << k << "] = " << result[k] << endl;

    cout << endl;

    if (N == 3)
    {
      // Demonstrate the traditional cross product too
      double x = vectors[0][0];
      double y = vectors[0][1];
      double z = vectors[0][2];

      double rhs_x = vectors[1][0];
      double rhs_y = vectors[1][1];
      double rhs_z = vectors[1][2];

      double cross_x = y * rhs_z - rhs_y * z;
      double cross_y = z * rhs_x - rhs_z * x;
      double cross_z = x * rhs_y - rhs_x * y;

      cout << cross_x << " " << cross_y << " " << cross_z << endl << endl;
    }

    return Vector_nD(result);
  }

\end{lstlisting}

Here we have the static dot product function:
\begin{lstlisting}

  static T dot_product(const Vector_nD<T, N>& a, const Vector_nD<T, N>& b)
  {
    return inner_product(
      a.components.begin(), 
      a.components.end(), 
      b.components.begin(), 0.0);
  }
};

\end{lstlisting}

Finally, we calculate the determinant of a square matrix using the cross and dot products as defined above:
\begin{lstlisting}

template <class T, typename size_t N>
T determinant_nxn(const MatrixX<T>& m)
{
  if (m.cols() != m.rows())
  {
    cout << "Matrix must be square" << endl;
    return 0;
  }

  // We will use this N-vector later, in the dot product operation
  Vector_nD<T, N> a_vector;

  for (size_t i = 0; i < N; i++)
    a_vector.components[i] = m(0, i);

  // We will use these (N - 1) N-vectors later, 
  // in the cross product operation
  vector<Vector_nD<T, N>> input_vectors;

  for (size_t i = 1; i < N; i++)
  {
    Vector_nD<T, N> b_vector;

    for (size_t j = 0; j < N; j++)
      b_vector.components[j] = m(i, j);

    input_vectors.push_back(b_vector);
  }

  // Compute the cross product using (N - 1) N-vectors
  Vector_nD<T, N> result = Vector_nD<T, N>::cross_product(input_vectors);

  // Compute the dot product
  T det = Vector_nD<T, N>::dot_product(a_vector, result);

  // These numbers should match
  cout << "Determinant:       " << det << endl;
  cout << "Eigen Determinant: " << m.determinant() << endl << endl;

  return det;
}

\end{lstlisting}

This main function is for testing the above code:
\begin{lstlisting}

int main(int argc, char** argv)
{
  srand(static_cast<unsigned int>(time(0)));

  const size_t N = 4; // Anything larger than 12 takes eons to solve for

  MatrixX<double> m(N, N);

  for (size_t i = 0; i < N; i++)
  {
    for (size_t j = 0; j < N; j++)
    {
      m(i, j) = rand() / static_cast<double>(RAND_MAX);

      if (rand() % 2 == 0)
        m(i, j) = -m(i, j);
    }
  }

  determinant_nxn<double, N>(m);

  return 0;
}


\end{lstlisting}



\section{Examples of the Claude cross product operator}

For $n = 3$:
\begin{eqnarray}
x_{0} &=& v_{01} v_{12} - v_{02} v_{11},\\
x_{1} &=& v_{00} v_{12} - v_{02} v_{10},\\
x_{2} &=& v_{00} v_{11} - v_{01} v_{10}.
\end{eqnarray}

For $n = 4$:
\begin{eqnarray}
x_{0} &=& 
   v_{01} v_{12} v_{23}
 - v_{01} v_{13} v_{22}
 - v_{02} v_{11} v_{23}
 + v_{02} v_{13} v_{21}
 + v_{03} v_{11} v_{22}
 - v_{03} v_{12} v_{21},\\
x_{1} &=& 
    v_{00} v_{12} v_{23}
  - v_{00} v_{13} v_{22}
 - v_{02} v_{10} v_{23}
 + v_{02} v_{13} v_{20}
  + v_{03} v_{10} v_{22}
  - v_{03} v_{12} v_{20},\\
x_{2} &=& 
v_{00} v_{11} v_{23}
- v_{00} v_{13} v_{21}
  - v_{01} v_{10} v_{23}
  + v_{01} v_{13} v_{20}
  + v_{03} v_{10} v_{21}
  - v_{03} v_{11} v_{20},\\
x_{3} &=& 
   v_{00} v_{11} v_{22}
 - v_{00} v_{12} v_{21}
 - v_{01} v_{10} v_{22}
 + v_{01} v_{12} v_{20}
 + v_{02} v_{10} v_{21}
 - v_{02} v_{11} v_{20}.
\end{eqnarray}


%\begin{thebibliography}{9}





%\bibitem{nasa} Williams. NASA Mercury Fact Sheet. (2024)



%\end{thebibliography}














\end{document}









