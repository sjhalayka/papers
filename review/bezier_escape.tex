\documentclass[12pt]{article}

\title{An axiomatic review of anisotropic quantum gravity}
\author{S. Halayka\footnote{sjhalayka@gmail.com}}
\date{\today\;\currenttime}

\usepackage{datetime}
\usepackage{listings}
\usepackage{cite}
\usepackage{xcolor}
\usepackage{graphicx}
\usepackage{setspace}
\usepackage{amsmath}
\usepackage{url}
\usepackage[margin=0.9in]{geometry}
%\doublespace



\begin{document}



 
\maketitle




\begin{abstract}
In Newton's and Einstein's theory, all mass gravitates in an {\textit{isotropic}} (spherical) manner.
In this paper, we will consider aspherical -- {\textit{anisotropic}} -- gravitating processes.
We discuss dark matter, as well as dark energy and the possibility of a final, $5$th interaction.
\end{abstract}





\section{Axioms}

Here we provide a list of $11$ axioms regarding gravitation.

\begin{enumerate}
\item The gravitational field is quantized into gravitons.
\item The gravitational field causes gravitational time dilation.
\item Speed causes kinematic time dilation.
\item Physical processes are interruptible, and are indeed interrupted when undergoing time dilation. 
\item Physical processes are computations.
\item Processes undergoing heavy time dilation due to speed are deflected twice as much as in Newtonian gravitation -- for neutrinos and photons, there is (practically) no internal process occurring to resist the gravitational attraction.
\item Computations can be optimized, so there is time contraction and length dilation to consider.
\item The number of gravitational degrees of freedom of a mass is finite.
\item There is no gravitational shadow -- the relaying of gravitons is the cause of gravitational time dilation.
\item Gravitationally-bound, pressure-free dusts, such as galaxies, have fractional dimension -- as the dimension reduces, the strength of the gravitation increases.
\item The self-optimization of the Universal process over time leads to length dilation, in the form of expansion -- the antithesis of attractive gravitation.
\end{enumerate}







\section{Results}

\begin{table}
\caption{Table of interactions, including a 5th interaction.}
\begin{center}
\begin{tabular}{| l | r | r |}
  \hline
  Type & Inherent spatial dimension & Communication spatial dimension \\
\hline
\hline
Gravitation (isotropic) & 3  & 4\\
Gravitation (oblate) & 2 & 3\\
Gravitation (prolate) & 1 & 2\\
Weak & 0 & 1\\
Electromagnetism & 1 & 0 \\
Strong & 2 & 1\\
$5$th interaction & 3 & 2 \\
  \hline
\end{tabular}
\end{center}
\end{table}
We have constructed this table by first taking into account the inherent 2-D nature of the strong interaction, and its 1-D communications (e.g. some Wilson lines).
Next, we extrapolate all the way up, to where isotropic gravitation is inherently 3-D, with 4-D communication (e.g. {\textit{the}} Wilson hypervolume). 
Finally, the possibility of a 5th interaction follows suit, in order to bring balance to the interactions in terms of their inherent spatial dimension.

Note that, unlike with the many Wilson lines per process, there is only one Wilson hypervolume, shared by all processes.
This means that gravitational interactions are {\textit{connectionless}} -- isotropic gravitation is {\textit{broadcast}}; there is no specific recipient (e.g. everyone is a target).
On the other hand, the strong interactions are more directed, and {\textit{connected}} -- strong interaction is {\textit{unicast}} or {\textit{multicast}}; there is a specific recipient (e.g. not everyone is a target).
For instance, the transition from broadcast transmission to directed transmission occurs as dark matter is factored in (e.g. as $1 < D < 3$).
Connectedness is an attribute of the non-gravitational interactions (e.g. weak, electromagnetic, strong, and 5th interaction).

Let's try to ask some good, if not elementary, questions:
\begin{itemize}
\item Does the Universe have exactly three inherent spatial dimensions?
If so, then is the Universe finite and closed (e.g. a 3-sphere in the Wilson hypervolume)?
\item Is a 5th interaction the same tetrahedral process that is predicted by (Wilson) loop quantum gravity \cite{loop}?
If so, then are superstring theory and loop quantum gravity fundamentally compatible?
\item Do gravitons undergo Shapiro time delay?
If so, then are graviton condensates naturally cold?
\item Is a photon a gas of constituent particles related to a 5th interaction?
\end{itemize}






\pagebreak







\begin{thebibliography}{9}

\bibitem{halayka} Halayka. Is the anisotropic interaction of luminous matter responsible for the extrinsic gravitation usually attributed to exotic dark matter? (2008)
\bibitem{halayka2} Halayka. A note on anisotropic quantum gravity. (2023)
\bibitem{wray} Wray. An Introduction to String Theory. (2011)
\bibitem{misner} Misner et al. Gravitation. (1970)
\bibitem{zuse} Zuse. Calculating Space. (1969)
\bibitem{wolfram} Wolfram. A New Kind of Science. (2002)
\bibitem{abrash} Abrash. Michael Abrash's Graphics Programming Black Book. (1997)
\bibitem{wainner} Wainner et al. The Book of Overclocking: Tweak Your PC to Unleash Its Power. (2003)
\bibitem{mcconnell} McConnell. Code Complete. 2E. (2004)
\bibitem{pikus} Pikus. The Art of Writing Efficient Programs: An advanced programmer's guide to efficient hardware utilization and compiler optimizations using C++ examples. (2021)
\bibitem{hooft} `t Hooft. Dimensional reduction in quantum gravity. (1993)
\bibitem{susskind} Susskind. The World as a Hologram. (1994)
\bibitem{bousso} Bousso. The holographic principle. (2002)
\bibitem{binney} Binney et al. Galactic Dynamics. Second Edition. (2008)
\bibitem{mandelbrot} Mandelbrot. The Fractal Geometry of Nature. (1982)
\bibitem{mm} Mitchell. Complexity: A Guided Tour. (2009)
\bibitem{loop} Gambini et al. A First Course in Loop Quantum Gravity (2011)








\end{thebibliography}





\end{document}









