\documentclass[12pt]{article}

\title{On the emission and absorption of long-range exchange particles }
\author{S. Halayka\footnote{sjhalayka@gmail.com}}
\date{\today}

\usepackage{listings}
\usepackage{cite}
\usepackage{xcolor}
\usepackage{graphicx}
\usepackage{setspace}
\usepackage{amsmath}
\usepackage{url}

\usepackage{caption}
\usepackage{subcaption}

%\usepackage[margin=1in]{geometry}
%\doublespace
\begin{document}




\maketitle

\begin{abstract}
Anisotropic emission of photons is considered.
It is found that the interaction strength increases as the photon emission goes from spherical, to circular, to beam-like.
\end{abstract}



\section{Dimensional reduction of the electromagnetic field}
In this paper we use Planck units, where $c = G = \hbar = k_B = \epsilon_0 = 1$.

Consider an isotropic photon emitter at the origin of 3D space $(0, 0, 0)$.
Also, consider a sphere-shaped photon absorber at position $(0, 100, 0)$, with a radius of $1$.

Numerically, it is found that the inverse normalized interaction strength is $41152.3$.
That is, when the photon emission goes from spherical (e.g. perfectly isotropic) to beam-like (e.g. perfectly anisotropic), the interaction strength increases by a factor of $41152.3$.
See Fig 1.
Similarly, when the photon emission is circular, the inverse normalized interaction strength is $315.259$.
See Fig 2.
The interaction strength varies for various absorber positions and radii.
Finally, when the photon emission is beam-like, the inverse normalized interaction strength is $1$ (by definition).
See Fig 3.

Altogether, these results are not surprising; it is all about the simple counting of the ray-sphere (e.g. photon-absorber) intersections.
The only assumption is that the temperature and the number of degrees of freedom are at least conserved, if not then increased, as the photon emitter goes from spherical, to circular, to beam-like.
It is otherwise axiomatic: anisotropic photon emitters increase in interaction strength.

Of course, for distances much larger than $100$, where the photons form what is practically a plane wave, it is possible to analytically obtain the inverse normalized interaction strength for a spherical photon emitter
\begin{equation}
I = \frac{ A_{{\textrm{sphere}}} }{ A_{{\textrm{circle}}} } = \frac{4 d^2}{r^2},
\end{equation}
where $d$ is the absorber distance from the origin, and $r$ is the absorber radius.
For a circular photon emitter, the corresponding equation is
\begin{equation}
I =  \frac{ L_{{\textrm{circle}}} }{ L_{{\textrm{line}}} }  =  \frac{\pi d}{r}.
\end{equation}
For a beam-like photon emitter:
\begin{equation}
I = 1.
\end{equation}

\section{Dimensional reduction of the gravitational field}

Can matter be coaxed into becoming an anisotropic graviton emitter?
Do we already see this in the Universe?
Is this the origin of the so-called dark matter in large-scale, gravitationally-bound systems?



\pagebreak





\begin{figure} 
\centering
\frame{  \includegraphics[width =1.0 in]{sphere_emitter.png}	}
  \caption{
Blue spherical emitter at $(0, 0, 0)$. 
Green absorber at $(0, 4, 0)$. 
Ray-sphere intersection locations are coloured in red.
A relatively low number of the photons emitted are absorbed.
Note that the interaction strength is isotropic.
}
\end{figure}


\begin{figure} 
\centering
\frame{  \includegraphics[width = 1.0 in]{circle_emitter.png}	}
  \caption{
Blue circular emitter at $(0, 0, 0)$. 
Green absorber at $(0, 4, 0)$. 
Ray-sphere intersection locations are coloured in red.
A relatively high number of the photons emitted are absorbed.
Note that the interaction strength orthogonal to the circle's plane reduces as the photon emitter goes from spherical to circular.
}
\end{figure}

\begin{figure} 
\centering
\frame{  \includegraphics[width = 1.0 in]{beam_emitter.png}	}
  \caption{
Blue beam emitter at $(0, 0, 0)$. 
Green absorber at $(0, 4, 0)$. 
Ray-sphere intersection location is coloured in red.
$100\%$ of the photons emitted are absorbed.
Note that the interaction strength for all angles other than head-on reduces as the photon emitter goes from spherical to beam-like.
}
\end{figure}






\end{document}









