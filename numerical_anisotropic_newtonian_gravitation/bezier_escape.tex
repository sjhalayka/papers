\documentclass[12pt]{article}

\title{On the Monte Carlo simulation of anisotropic Newtonian gravitation}
\author{S. Halayka\footnote{sjhalayka@gmail.com}}
\date{\today\;\currenttime}

\usepackage{datetime}
\usepackage{listings}
\usepackage{cite}
\usepackage{xcolor}
\usepackage{graphicx}
\usepackage{setspace}
\usepackage{amsmath}
\usepackage{url}
\usepackage[margin=0.8in]{geometry}
\usepackage{listings}


\usepackage{xcolor}
\lstset { %
    language=C++,
    backgroundcolor=\color{black!5}, % set backgroundcolor
    basicstyle=\footnotesize,% basic font setting
    showstringspaces=false,
}


%\doublespace

%\usepackage[]{lineno}
%\linenumbers


\begin{document}



 
\maketitle

\begin{abstract}
This paper contains a short introduction to anisotropic Newtonian gravitation.
\end{abstract}



\section{Introduction}

First see \cite{halayka} for a short tutorial for C++ programmers on Newtonian gravitation.

This paper introduces a Monte Carlo method for generating anisotropic gravitational fields.

\section{Brute force: field line intersection density gradient}

Regarding the holographic principle \cite{hooft, susskind}, the number of gravitational field lines for a black hole is related to the event horizon area $A$:
\begin{equation}
n = \frac{A k c^3}{ 4 G \hbar \log 2}.
\end{equation}
For a Schwarzschild black hole in particular \cite{misner}, the event horizon radius $r_s$ is
\begin{equation}
r_s = \sqrt{\frac{A}{4 \pi}} = \sqrt{\frac{n G \hbar \log 2}{k c^3 \pi}},
\end{equation}
and the mass of the Schwarzschild black hole is
\begin{equation}
M = \frac{c^2 r_s}{2 G} = \sqrt{\frac{n c \hbar \log 2}{4 G k \pi}}.
\end{equation}

Where the $\beta$ function is the integer field line collision count:
\begin{equation}
\alpha = \frac{\beta(R + \epsilon) - \beta(R)}{\epsilon}.
\end{equation}
\begin{equation}
g = \frac{-\alpha}{r^2}. 
\end{equation}

The Newtonian gravitational variables are:
\begin{equation}
a_N = \frac{G M}{R^2} = \sqrt{\frac{n G c \hbar \log 2}{4 k \pi R^4}},
\end{equation}
\begin{equation}
v_N = \sqrt{a_N R}.
\end{equation}
\begin{equation}
g_N = \frac{a_N k 2 \pi M}{R c \hbar \log 2}. 
\end{equation}

The flat rotation curve variables are:
\begin{equation}
v_{\textrm{flat}} = x v_N
\end{equation}
where $x = 2$ for example.
\begin{equation}
a_{\textrm{flat}} = \frac{v_{\textrm{flat}}^2}{R} = \frac{g R c \hbar \log 2}{k 2 \pi M}.
\end{equation}




\begin{equation}
a_{\textrm{flat}} \propto g.
\end{equation}
\begin{equation}
a_{\textrm{ratio}} = \frac{a_{\textrm{flat}}}{a_N}. 
\end{equation}
\begin{equation}
g_{\textrm{ratio}} = \frac{g}{g_N}. 
\end{equation}

To find $D$, look for where $g_{\textrm{ratio}} \geq a_{\textrm{ratio}}$, starting from $D = 3$, marching toward $D = 2$.






\begin{figure} 
\centering
\label{fig3}
  \includegraphics[width = 1.5 in]{3.png}
  \caption{
Where $D = 3$, as viewed from the side.
The field lines are isotropic, spherical.
}
\end{figure}
\begin{figure} 
\centering
\label{fig4}
  \includegraphics[width = 1.5 in]{2.1.png}
  \caption{
Where $D = 2.1$, as viewed from the side.
The field lines are increasingly anisotropic.
}
\end{figure}
\begin{figure} 
\centering
\label{fig5}
  \includegraphics[width = 1.5 in]{2.001.png}
  \caption{
Where $D = 2.001$, as viewed from the side.
The field lines are anisotropic, disk-like.
}
\end{figure}





\begin{figure} 
\centering
\label{fig3}
  \includegraphics[width = 7 in]{numerical_versus_analytical.png}
  \caption{
$R = 100$, $r = 1$, $n = 10^{8}$, $\epsilon = 1$.
The analytical plot is generated by the formula $y = n / (2 R^D)$.
}
\end{figure}


\begin{figure} 
\centering
\label{fig6}
  \includegraphics[width = 7 in]{doppler.png}
  \caption{
Visualization of the relativistic Doppler effect for $10^6$ stars.
The Newtonian orbit is on the left, and the flat rotation curve orbit is on the right.
The redshift of the wavelength indicates stars moving away from the camera, and blueshift of the wavelength indicates stars moving toward the camera.
Thus, the stars in the galaxy are orbiting counterclockwise.
On the left, the wavelength is dependent on angle and distance from the galactic centre.
On the right, the wavelength is dependent only on angle, which means that there is a constant orbit speed that is independent of the distance from the galactic centre.
This is exactly what Vera Rubin discovered in the disks of the galaxies that she observed, and so dark matter was posited.
}
\end{figure}





\begin{thebibliography}{9}

\bibitem{halayka} Halayka. Newtonian gravitation from scratch, for C++ programmers. (2024)
\bibitem{hooft} `t Hooft. Dimensional reduction in quantum gravity. (1993)
\bibitem{susskind} Susskind. The World as a Hologram. (1994)
\bibitem{misner} Misner et al. Gravitation. (1970)

\end{thebibliography}














\end{document}









