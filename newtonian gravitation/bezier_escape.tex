\documentclass[12pt]{article}

\title{Newtonian gravitation for C++ programmers}
\author{S. Halayka\footnote{sjhalayka@gmail.com}}
\date{\today\;\currenttime}

\usepackage{datetime}
\usepackage{listings}
\usepackage{cite}
\usepackage{xcolor}
\usepackage{graphicx}
\usepackage{setspace}
\usepackage{amsmath}
\usepackage{url}
\usepackage[margin=1.0in]{geometry}
\usepackage{listings}


\usepackage{xcolor}
\lstset { %
    language=C++,
    backgroundcolor=\color{black!5}, % set backgroundcolor
    basicstyle=\footnotesize,% basic font setting
    showstringspaces=false,
}


%\doublespace

%\usepackage[]{lineno}
%\linenumbers


\begin{document}



 
\maketitle

\begin{abstract}
...
\end{abstract}


\section{Numerical: integer field line count}



\begin{lstlisting}
long long unsigned int get_intersecting_line_count(
	const vector<vector_3>& unit_vectors,
	const vector_3 sphere_location,
	const real_type sphere_radius)
{
	long long unsigned int count = 0;

	vector_3 cross_section_edge_dir(sphere_location.x, sphere_radius, 0);
	cross_section_edge_dir.normalize();

	vector_3 receiver_dir(sphere_location.x, 0, 0);
	receiver_dir.normalize();

	const real_type min_dot = cross_section_edge_dir.dot(receiver_dir);

	for (size_t i = 0; i < unit_vectors.size(); i++)
		if (unit_vectors[i].dot(receiver_dir) >= min_dot)
			count++;

	return count;
}
\end{lstlisting}



\begin{lstlisting}
int main(int argc, char** argv)
{
	// Field line count
	const size_t n = 1000000000;

	cout << "Allocating memory for field lines" << endl;
	vector<vector_3> unit_vectors(n);

	for (size_t i = 0; i < n; i++)
	{
		unit_vectors[i] = RandomUnitVector();

		static const size_t output_mod = 10000;

		if (i % output_mod == 0)
			cout << "Getting pseudorandom locations: " 
			<< static_cast<float>(i) / n << endl;
	}

	string filename = "newton.txt";
	ofstream out_file(filename.c_str());
	out_file << setprecision(30);

	const real_type start_distance = 10.0;
	const real_type end_distance = 100.0;
	const size_t distance_res = 1000.0;

	const real_type distance_step_size = 
		(end_distance - start_distance) 
		/ (distance_res - 1);

	for (size_t step_index = 0; step_index < distance_res; step_index++)
	{
		const real_type r = 
			start_distance + 
			step_index * distance_step_size;

		const vector_3 receiver_pos(r, 0, 0);
		const real_type receiver_radius = 1.0;

		const real_type epsilon = 1.0;

		vector_3 receiver_pos_plus = receiver_pos;
		receiver_pos_plus.x += epsilon;

		const long long signed int collision_count_plus = 
			get_intersecting_line_count(
				unit_vectors, 
				receiver_pos_plus, 
				receiver_radius);
		
		const long long signed int collision_count = 
			get_intersecting_line_count(
				unit_vectors, 
				receiver_pos, 
				receiver_radius);
		
		const real_type gradient = 
			static_cast<real_type>
			(collision_count_plus - collision_count) 
			/ epsilon;
		
		const real_type gradient_strength = 
			-gradient 
			/ (receiver_radius * receiver_radius);

		cout << "r: " << r << " gradient strength: " 
		<< gradient_strength << endl;

		out_file << r << " " << gradient_strength << endl;
	}

	out_file.close();

	return 0;
}
\end{lstlisting}





\section{Analytical: real field line count}


\begin{lstlisting}
real_type get_intersecting_line_count(
	const real_type n,
	const vector_3 sphere_location,
	const real_type sphere_radius)
{
	const real_type big_area = 
		4 * pi * sphere_location.x * sphere_location.x;

	const real_type small_area = 
		pi * sphere_radius * sphere_radius;
	
	const real_type ratio = 
		small_area / big_area;
	
	return n * ratio;
}
\end{lstlisting}


\begin{lstlisting}
int main(int argc, char** argv)
{
	const real_type emitter_radius = 1.0;
	
	const real_type emitter_area = 
		4.0 * pi * emitter_radius * emitter_radius;

	// Field line count
	// re: holographic principle:
	const real_type n = 
		(c3 * emitter_area) 
		/ (log(2.0) * 4.0 * G * hbar);
	
	const real_type emitter_mass = c2 * emitter_radius / (2.0 * G);

	// 1.73502e+70 is the 't Hooft-Susskind constant:
	// the number of field lines for a black hole of
	// unit Schwarzschild radius
	//
	//const real_type G_ = 
	//	(c3 * pi) 
	//	/ (log(2.0) * hbar * 1.73502e+70);

	const string filename = "newton.txt";
	ofstream out_file(filename.c_str());
	out_file << setprecision(30);

	const real_type start_distance = 10.0;
	const real_type end_distance = 100.0;
	const size_t distance_res = 1000;

	const real_type distance_step_size =
		(end_distance - start_distance)
		/ (distance_res - 1);

	for (size_t step_index = 0; step_index < distance_res; step_index++)
	{
		const real_type r =
			start_distance + step_index * distance_step_size;

		const vector_3 receiver_pos(r, 0, 0);
		const real_type receiver_radius = 1.0;

		const real_type epsilon = 1.0;

		vector_3 receiver_pos_plus = receiver_pos;
		receiver_pos_plus.x += epsilon;

		// https://en.wikipedia.org/wiki/Directional_derivative
		const real_type collision_count_plus =
			get_intersecting_line_count(
				n,
				receiver_pos_plus,
				receiver_radius);

		const real_type collision_count =
			get_intersecting_line_count(
				n,
				receiver_pos,
				receiver_radius);

		const real_type gradient =
			(collision_count_plus - collision_count)
			/ epsilon;

		real_type gradient_strength =
			-gradient
			/ (receiver_radius * receiver_radius);

		const real_type gradient_strength_ = 
			n / (2.0 * pow(receiver_pos.x, 3.0));

		const real_type newton_strength = 
			n * c * hbar * log(2.0)
			/ 
			(pow(receiver_pos.x, 2.0) 
				* emitter_mass * 4.0 * pi);

		const real_type newton_strength_ =
			c4 * emitter_area
			/ (16.0 * pi * G 
				* pow(receiver_pos.x, 2.0) * emitter_mass);

		const real_type newton_strength__ =
			G * emitter_mass / pow(receiver_pos.x, 2.0);

		//cout << newton_strength__ / newton_strength << endl;

		cout << "r: " << r << " gradient strength: "
			<< gradient_strength << endl;

		out_file << r << " " << gradient_strength << endl;
	}

	out_file.close();

	return 0;
}
\end{lstlisting}



%\begin{thebibliography}{9}




%\end{thebibliography}






\end{document}









