\documentclass[12pt]{article}

\title{Investigating the plasticity of spacetime}
\author{S. Halayka\footnote{sjhalayka@gmail.com}}
\date{\today\;\currenttime}

\usepackage{datetime}
\usepackage{listings}
\usepackage{cite}
\usepackage{xcolor}
\usepackage{graphicx}
\usepackage{setspace}
\usepackage{amsmath}
\usepackage{url}
\usepackage[margin=0.8in]{geometry}
\usepackage{listings}
\usepackage{derivative}


\usepackage{xcolor}
\lstset { %
    language=C++,
    backgroundcolor=\color{black!5}, % set backgroundcolor
    basicstyle=\footnotesize,% basic font setting
    showstringspaces=false,
}


%\doublespace

%\usepackage[]{lineno}
%\linenumbers


\begin{document}



 
\maketitle

\begin{abstract}
By considering plasticity as an attribute of spacetime, an explanation for the interference pattern in the quantum double-slit experiment is presented.
\end{abstract}




\section{Introduction}

In the quantum double-slit experiment, an interference pattern is
produced even when the test bodies are sent through the apparatus one
at a time.

The Copenhagen interpretation explains this phenomenon by considering
the test body as a self-interfering wave that takes all paths within
the apparatus at the same time. Only upon interaction with other
focused energy does the test body collapse into a point-like state,
forcing it to take a single path. Examples of such interaction are
when a photon is sent to probe the location of the test body during
its travel, or when the test body reaches the end of the apparatus and
impinges upon the material used to visualize the results of the
experiment.

In order to provide further support for the Copenhagen interpretation
of self-interference, it could be verified that the spacetime region
through which the test body travels is not plastic in nature at the
macroscopic scale.


\section{Proposed experiment}
In-line with current methods, the apparatus consists of a closed
container that has been evacuated of all extraneous matter. At one end
is the test body emitter, and on the other end is the visualization
material. Placed somewhere between these ends is a barrier containing
two thin slits that allow the test bodies to travel through.

No attempt should be made to verify which of the two slits the test
body traveled through.

In between the emission of each test body, a portion of the closed
container is flooded with matter (ex: atmospheric gases), and then re-
evacuated. The flood matter should not come into contact with the
visualization material, in order to avoid over-exposure.

By exposing a portion of the closed container to the stochastic
motions of the flood matter, any preferential path formed by the
plasticity of the spacetime should be nullified.

If this method causes no observable change in the pattern produced on
the visualization material after sufficient reiteration, then it may
be interpreted that the plasticity of spacetime is not a critical
factor in the paths followed by the test bodies within the apparatus.



\section{Conclusion}

A macroscopic equivalent for spacetime plasticity can be envisioned by
considering the state of a snowy ski hill before and after a day of
heavy usage. Conscious decision aside, the majority of paths taken
over time by the skiers will overlap and group toward the centre of
the hill because it is easier to travel through arteries of flatly
packed snow.

Does the spacetime region within the apparatus exhibit this type of
plasticity? If so, do the test bodies also prefer to take (or avoid)
the path well-traveled?





\end{document}









