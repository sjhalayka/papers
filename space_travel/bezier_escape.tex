\documentclass[12pt]{article}

\title{On ultra-high-speed interstellar travel}
\author{S. Halayka\footnote{sjhalayka@gmail.com}}
\date{\today}

\usepackage{listings}
\usepackage{cite}
\usepackage{xcolor}
\usepackage{graphicx}
\usepackage{setspace}
\usepackage{amsmath}
\usepackage{url}
\usepackage[margin=1in]{geometry}
\begin{document}



 
\maketitle

\begin{abstract}
The possibility of ultra-high-speed interstellar travel is considered.
The role of the Devil's advocate is played, purely for the sake of interest -- it is considered that ultra-high-speed interstellar travel is impossible.
\end{abstract}



\section{Introduction}
This is the theory of processes, where external interaction results in internal time dilation, which in extreme cases can result in the practically irreversible \textit{disruption} of said processes.
This is to say that the relaying and reflection of the externally-generated exchange particles causes the process formed by internally-generated exchange particles to reduce in frequency. 
When we say frequency, we mean how often something happens over some macroscopic amount of time, not necessarily how fast something cycles.

The biggest difference between the view given in this paper and the contemporary view is that in this view, the process overcome by externally-generated exchange particles is \textit{forgotten}.

In this paper we use Planck units, where $c = G = \hbar = k = 1$.
As such, these constants are dropped from the equations in this paper.




\section{On the fate of the Devil's clocks}
	If we put an atomic clock in a deep gravity well, and then remove it to some large distance away from the event horizon, the clock will no longer be functioning properly.
Also, if we accelerate a clock to ultra-high-speeds, and then reduce its speed considerably, the clock will no longer be functioning properly.
The internal process is all but forgotten, as the internal process is \textit{assimilated} by the external process.
The entropy gained during the assimilation must be preserved in order to comply with the second law of thermodynamics (the change in entropy over the change in time must be non-negative) \cite{jacobson}:
\begin{equation}
dS/dt \geq 0.
\end{equation}
Entropy is the measure of external process -- of external \textit{connectivity}.




\section{Random integer, quantum chance}
Consider that the internal process only occurs at random intervals, some longer than others, not necessarily cyclically -- the cycle rate is an upper bound on how often it happens.

The time rate $dt$ in a gravitational well that surrounds a rotating, electrically-neutral (Kerr) black hole \cite{mtw} is:
\begin{equation}
dt = \sqrt{1 - \frac{2Mr}{r^2 + (J/M)^2 \cos^2 \theta}} \in (0, 1] ,
\end{equation}
where $M$ is the black hole mass, $J$ is the black hole angular momentum, and $\theta$ is the angle of incidence.
Also, $dt > 0$, which means that $r$ lies outside of the black hole event horizon.

Similarly, for an accelerated body, where $v$ is the speed, and $v < 1$:
\begin{equation}
dt = \sqrt{1 - {v^2}}  \in (0, 1].
\end{equation}

Note that $dt$ is also the \textit{probability} that internal process occurs.
Accordingly, $1 - dt$ is the probability that external process occurs.

The characteristic wavelength $\lambda$ is an integer, where the Planck length is $\ell_P = 1$:
\begin{equation}
\lambda = \lfloor{1}/{dt} \rfloor \in  [1, 2, 3, ..., \infty) ,
\end{equation}
and $\alpha$ is a random integer that has an upper limit of some non-zero multiple of $\lambda$:
\begin{equation}
x = \alpha \textrm{ mod } \lambda \in [0, 1, 2, ..., \lambda).
\end{equation}
If $x = 0$, then the internal process occurs by chance, else the external process occurs \cite{dilbert}.




\section{Training the one process}
Consider the notion of \textit{training} a process via slow, non-jerky acceleration -- training the process to propagate faster and faster (that is, to move more and more in one particular direction).
Once we train a process, the old process that it was is no longer.
\textit{There's only one process.}
This is unlike the contemporary view where the process can be a superposition of two processes (one propagating through space fast, and one propagating through time slow), where we can simply untrain/retrain the process by using propulsion to slow down.
In reality, the old process is forgotten, and braking doesn't make it come back, if the thrusters are even operational anymore.

It's true that the newly-trained process will always contain the remnants of the old process, since the speed of light is not attainable, but those remnants being converted back to the way the process was before its interstellar trip is a matter of pure luck.
The training is practically irreversible in a quantum model, where $dt \approx 0$ -- the gain in entropy must be preserved.






\section{On being part of a black hole}
To be part of a black hole, where $dt = 0$, a body's internal process is minimized, and its external process is maximized.
Perfect, total external connectivity.




\section{Conclusion}
Where $dt \approx 0$, the chances are extremely slim that internal process occurs.
In effect, ultra-high-speed interstellar travel is not possible.
The higher the speed, the lesser the chance.
The process is permanently \textit{interrupted}.

If Nature is quantized and random and potentially highly irregular as such where $dt \approx 0$, then the timing of the internal process would go out of whack, and the internal process would be forgotten without all of the redundancy and error-correction that otherwise occurs in the classical, Newtonian limit where $dt \approx 1$.

Quantum effects should start to be noticeable where $dt = 0.5$, and $v = \sqrt{3} / 2 = \sin(\pi/3) = 0.866025$.

The maximum entropy for an accelerated Kerr black hole of any speed $v < 1$ is assumed to be the Bekenstein-Hawking entropy \cite{jacobson}, which is the black hole event horizon area divided by $4$:
\begin{equation}
S_{BH} = 2 \pi M \left(M + \sqrt{M^2 - (J/M)^2}\right) \in (0, 4\pi M^2).
\end{equation}
The entropy is independent of the black hole's speed because the black hole is all about external process, all the time -- the entropy is independent of how \textit{focused} this perfect external process is in one particular direction.
Otherwise, for non-black holes, it's assumed that the entropy can never reach $S_{BH}$, because the speed of light is not attainable -- the process can never be perfectly focused in one particular direction.
When we say focused, we mean alignment of the oscillations into one direction -- to be \textit{anisotropic}.
Consider a toy model, where $n$ is the number of fully uncoupled asymmetric oscillators comprising the non-black hole.
The total dot product $y$ is:
\begin{equation}
y = \sum_{i = 1}^{n - 1}  \sum_{j = i + 1}^{n} {\vec{A}}_i  \cdot {\vec{A}}_j.
\end{equation}
The 3D vectors ${\vec{A}}$ each have a length $\ell \in (0, 1)$, which represents an oscillator's level of asymmetry -- an oscillator's degree of being external in one particular direction.
The total number of dot product operations $z$ is:
\begin{equation}
z = \frac{(n - 1)n}{2}.
\end{equation}
The focus $\omega$ (anisotropy) is the normalized average dot product:
\begin{equation}
\omega = \frac{1 + {y}/{z}}{2} \in [0, 1).
\end{equation}
The entropy is:
\begin{equation}
S = \omega S_{BH} \in [0,  S_{BH}).
\end{equation}



\pagebreak


\begin{thebibliography}{9}
\bibitem{jacobson} Jacobson. (1996) ``Introductory Lectures on Black Hole Thermodynamics''
\bibitem{mtw} Misner, Thorne, Wheeler. (1973) ``Gravitation''
\bibitem{dilbert} Adams. (2001) ``Tour of Accounting'' -- \url{https://dilbert.com/strip/2001-10-25}
\end{thebibliography}



\end{document}









